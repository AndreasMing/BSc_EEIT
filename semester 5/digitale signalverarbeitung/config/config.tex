\usepackage{amssymb, amsmath} % includes some neato math environments, symbols and stuff
\numberwithin{equation}{section}

\usepackage[utf8]{inputenc} % correctly encodes the formatting of the text files (since we live in the future of UTF8)
\usepackage{lastpage} % counts all the pages, used in footer
\usepackage{hyperref}

% Flexible Multicolumn Option
\usepackage{multicol}
\setlength\columnsep{20pt}

% Package for enabling colors (colorful output)
\usepackage[dvipsnames]{xcolor}
\definecolor{darkgreen}{HTML}{014f32}

% Icons - http://mirrors.ctan.org/fonts/fontawesome5/doc/fontawesome5.pdf
% This one is neat, when you want to add some icons
\usepackage{fontawesome5}

% Depending on the type of document, mostly summaries, I remove the spacing Display-styled Equations insert, since it's a waste of space.
\usepackage[nodisplayskipstretch]{setspace}

% Font Configuration
\usepackage{lmodern}
% \renewcommand{\familydefault}{\sfdefault}
%\usepackage{cmbright}
\usepackage[scaled=0.85]{beramono} % monospaced font (for code blocks)

% Heading
\usepackage{fancyhdr}

\renewcommand{\headrulewidth}{1pt}
\renewcommand{\footrulewidth}{1pt}

\pagestyle{fancy}
\fancyhead{} % clear all header fields

\ifcsname subtitle\endcsname
  \fancyhead[R]{\myTitle\space -- \subtitle}%
\else
  \fancyhead[R]{\myTitle}%
\fi
\fancyhead[L]{HSLU T\&A}
\fancyfoot{} % clear all footer fields
\fancyfoot[C]{\thepage\ / \pageref{LastPage}}
\fancyfoot[R]{DSVB}
\fancyfoot[L]{\today}




% introduces conditions environment to create nice equation parameter description
\usepackage{array}
\newenvironment{conditions}
  {\par\vspace{\abovedisplayskip}\noindent\begin{tabular}{>{$}l<{$} @{${}:{}$} l}}
  {\end{tabular}\par\vspace{\belowdisplayskip}}

% Used for pandoc code block generations.
% Introduces breaklines for overflowing code blocks, by defining the Highlighting environment with breakline & symbol.
% Don't know what commandchars does :)
\usepackage{fvextra}
\DefineVerbatimEnvironment{Highlighting}{Verbatim}{
  breaklines,
  breaksymbolleft={\textcolor{gray}{\scriptsize\ensuremath\hookrightarrow}},
  commandchars=\\\{\}
}

% stylized section headings!
\let\paragraph\oldparagraph
\let\subparagraph\oldsubparagraph

\usepackage{xhfill}
\usepackage[explicit]{titlesec}
\renewcommand{\paragraph}[1]{\oldparagraph{#1}\mbox{}\par}

\titleformat{\section}[block]{\bfseries\Large}{\thetitle.}{3mm}{#1\space\xrfill[0.6ex]{1pt}}

% Math notations
\usepackage{amsfonts}

% TIKZ can be used to create many different types of diagramms, drawings, etc.
%\usepackage{tikz}
%\newcommand*\circled[1]{\tikz[baseline=(char.base)]{
%            \node[shape=circle,draw,inner sep=2pt] (char) {#1};}}
%\usetikzlibrary{shapes,arrows,arrows.meta,matrix}