\usepackage{amssymb, amsmath}
\numberwithin{equation}{section}

\usepackage[utf8]{inputenc}
\usepackage{lastpage} % counts all the pages, used in footer
\usepackage{hyperref}

% Flexible Multicolumn Option
\usepackage{multicol}
\setlength\columnsep{20pt}

% Package for enabling colors (colorful output)
\usepackage[dvipsnames]{xcolor}
\definecolor{darkgreen}{HTML}{014f32}

% Icons - http://mirrors.ctan.org/fonts/fontawesome5/doc/fontawesome5.pdf
\usepackage{fontawesome5}


\usepackage{tabularx}
\renewcommand\tabularxcolumn[1]{m{#1}} % vertically center content


\usepackage[nodisplayskipstretch]{setspace}

% Font Configuration
\usepackage{lmodern}
\renewcommand{\familydefault}{\sfdefault}
\usepackage{cmbright}
\usepackage[scaled=0.85]{beramono}

% 
\usepackage{fancyhdr}

\renewcommand{\headrulewidth}{1pt}
\renewcommand{\footrulewidth}{1pt}

\pagestyle{fancy}
\fancyhead{} % clear all header fields
\makeatletter
\fancyhead[R]{\@title}
\makeatother
\fancyhead[L]{HSLU T\&A}
\fancyfoot{} % clear all footer fields
\fancyfoot[C]{\thepage\ / \pageref{LastPage}}
\fancyfoot[R]{NRT}
\fancyfoot[L]{\today}

% introduces conditions environment to create nice equation parameter descriptions
\usepackage{array}

\newenvironment{conditions}
  {\par\vspace{\abovedisplayskip}\noindent\begin{tabular}{>{$}l<{$} @{${}:{}$} l}}
  {\end{tabular}\par\vspace{\belowdisplayskip}}

% Used for pandoc code block generations.
% Introduces breaklines for overflowing code blocks, by defining the Highlighting environment with breakline & symbol.
% Don't know what commandchars does :)
\usepackage{fvextra}
\DefineVerbatimEnvironment{Highlighting}{Verbatim}{
  breaklines,
  breaksymbolleft={\textcolor{gray}{\scriptsize\ensuremath\hookrightarrow}},
  commandchars=\\\{\}
}



\let\paragraph\oldparagraph
\let\subparagraph\oldsubparagraph

\usepackage{xhfill}
\usepackage[explicit]{titlesec}
\renewcommand{\paragraph}[1]{\oldparagraph{#1}\mbox{}\par}

\titleformat{\section}[block]{\bfseries\Large}{\thetitle.}{3mm}{#1\space\xrfill[0.6ex]{1pt}}


\usepackage{tikz}
\newcommand*\circled[1]{\tikz[baseline=(char.base)]{
            \node[shape=circle,draw,inner sep=2pt] (char) {#1};}}
            
\usetikzlibrary{shapes,arrows,arrows.meta,matrix}