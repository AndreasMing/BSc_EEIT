% Options for packages loaded elsewhere
\PassOptionsToPackage{unicode}{hyperref}
\PassOptionsToPackage{hyphens}{url}
\PassOptionsToPackage{dvipsnames,svgnames,x11names}{xcolor}
%
\documentclass[
  10pt,
  a4paper,
]{article}

\usepackage{amsmath,amssymb}
\usepackage{lmodern}
\usepackage{iftex}
\ifPDFTeX
  \usepackage[T1]{fontenc}
  \usepackage[utf8]{inputenc}
  \usepackage{textcomp} % provide euro and other symbols
\else % if luatex or xetex
  \usepackage{unicode-math}
  \defaultfontfeatures{Scale=MatchLowercase}
  \defaultfontfeatures[\rmfamily]{Ligatures=TeX,Scale=1}
\fi
% Use upquote if available, for straight quotes in verbatim environments
\IfFileExists{upquote.sty}{\usepackage{upquote}}{}
\IfFileExists{microtype.sty}{% use microtype if available
  \usepackage[]{microtype}
  \UseMicrotypeSet[protrusion]{basicmath} % disable protrusion for tt fonts
}{}
\makeatletter
\@ifundefined{KOMAClassName}{% if non-KOMA class
  \IfFileExists{parskip.sty}{%
    \usepackage{parskip}
  }{% else
    \setlength{\parindent}{0pt}
    \setlength{\parskip}{6pt plus 2pt minus 1pt}}
}{% if KOMA class
  \KOMAoptions{parskip=half}}
\makeatother
\usepackage{xcolor}
\usepackage[top=25mm,bottom=25mm,left=25mm,right=25mm]{geometry}
\setlength{\emergencystretch}{3em} % prevent overfull lines
\setcounter{secnumdepth}{3}
% Make \paragraph and \subparagraph free-standing
\ifx\paragraph\undefined\else
  \let\oldparagraph\paragraph
  \renewcommand{\paragraph}[1]{\oldparagraph{#1}\mbox{}}
\fi
\ifx\subparagraph\undefined\else
  \let\oldsubparagraph\subparagraph
  \renewcommand{\subparagraph}[1]{\oldsubparagraph{#1}\mbox{}}
\fi


\providecommand{\tightlist}{%
  \setlength{\itemsep}{0pt}\setlength{\parskip}{0pt}}\usepackage{longtable,booktabs,array}
\usepackage{calc} % for calculating minipage widths
% Correct order of tables after \paragraph or \subparagraph
\usepackage{etoolbox}
\makeatletter
\patchcmd\longtable{\par}{\if@noskipsec\mbox{}\fi\par}{}{}
\makeatother
% Allow footnotes in longtable head/foot
\IfFileExists{footnotehyper.sty}{\usepackage{footnotehyper}}{\usepackage{footnote}}
\makesavenoteenv{longtable}
\usepackage{graphicx}
\makeatletter
\def\maxwidth{\ifdim\Gin@nat@width>\linewidth\linewidth\else\Gin@nat@width\fi}
\def\maxheight{\ifdim\Gin@nat@height>\textheight\textheight\else\Gin@nat@height\fi}
\makeatother
% Scale images if necessary, so that they will not overflow the page
% margins by default, and it is still possible to overwrite the defaults
% using explicit options in \includegraphics[width, height, ...]{}
\setkeys{Gin}{width=\maxwidth,height=\maxheight,keepaspectratio}
% Set default figure placement to htbp
\makeatletter
\def\fps@figure{htbp}
\makeatother

\usepackage{amssymb, amsmath}
\numberwithin{equation}{section}

\usepackage[utf8]{inputenc}
\usepackage{lastpage} % counts all the pages, used in footer
\usepackage{hyperref}

% Flexible Multicolumn Option
\usepackage{multicol}
\setlength\columnsep{20pt}

% Package for enabling colors (colorful output)
\usepackage[dvipsnames]{xcolor}
\definecolor{darkgreen}{HTML}{014f32}

% Icons - http://mirrors.ctan.org/fonts/fontawesome5/doc/fontawesome5.pdf
\usepackage{fontawesome5}


\usepackage{tabularx}
\renewcommand\tabularxcolumn[1]{m{#1}} % vertically center content


\usepackage[nodisplayskipstretch]{setspace}

% Font Configuration
\usepackage{lmodern}
\renewcommand{\familydefault}{\sfdefault}
\usepackage{cmbright}
\usepackage[scaled=0.85]{beramono}

% 
\usepackage{fancyhdr}

\renewcommand{\headrulewidth}{1pt}
\renewcommand{\footrulewidth}{1pt}

\pagestyle{fancy}
\fancyhead{} % clear all header fields
\makeatletter
\fancyhead[R]{\@title}
\makeatother
\fancyhead[L]{HSLU T\&A}
\fancyfoot{} % clear all footer fields
\fancyfoot[C]{\thepage\ / \pageref{LastPage}}
\fancyfoot[R]{AEMBS}
\fancyfoot[L]{\today}

% introduces conditions environment to create nice equation parameter descriptions
\usepackage{array}

\newenvironment{conditions}
  {\par\vspace{\abovedisplayskip}\noindent\begin{tabular}{>{$}l<{$} @{${}:{}$} l}}
  {\end{tabular}\par\vspace{\belowdisplayskip}}

% Used for pandoc code block generations.
% Introduces breaklines for overflowing code blocks, by defining the Highlighting environment with breakline & symbol.
% Don't know what commandchars does :)
\usepackage{fvextra}
\DefineVerbatimEnvironment{Highlighting}{Verbatim}{
  breaklines,
  breaksymbolleft={\textcolor{gray}{\scriptsize\ensuremath\hookrightarrow}},
  commandchars=\\\{\}
}
\makeatletter
\makeatother
\makeatletter
\makeatother
\makeatletter
\@ifpackageloaded{caption}{}{\usepackage{caption}}
\AtBeginDocument{%
\ifdefined\contentsname
  \renewcommand*\contentsname{Inhaltsverzeichnis}
\else
  \newcommand\contentsname{Inhaltsverzeichnis}
\fi
\ifdefined\listfigurename
  \renewcommand*\listfigurename{Abbildungsverzeichnis}
\else
  \newcommand\listfigurename{Abbildungsverzeichnis}
\fi
\ifdefined\listtablename
  \renewcommand*\listtablename{Tabellenverzeichnis}
\else
  \newcommand\listtablename{Tabellenverzeichnis}
\fi
\ifdefined\figurename
  \renewcommand*\figurename{Abbildung}
\else
  \newcommand\figurename{Abbildung}
\fi
\ifdefined\tablename
  \renewcommand*\tablename{Tabelle}
\else
  \newcommand\tablename{Tabelle}
\fi
}
\@ifpackageloaded{float}{}{\usepackage{float}}
\floatstyle{ruled}
\@ifundefined{c@chapter}{\newfloat{codelisting}{h}{lop}}{\newfloat{codelisting}{h}{lop}[chapter]}
\floatname{codelisting}{Listing}
\newcommand*\listoflistings{\listof{codelisting}{Listingverzeichnis}}
\makeatother
\makeatletter
\@ifpackageloaded{caption}{}{\usepackage{caption}}
\@ifpackageloaded{subcaption}{}{\usepackage{subcaption}}
\makeatother
\makeatletter
\@ifpackageloaded{tcolorbox}{}{\usepackage[many]{tcolorbox}}
\makeatother
\makeatletter
\@ifundefined{shadecolor}{\definecolor{shadecolor}{HTML}{f7f7f7}}
\makeatother
\makeatletter
\makeatother
\ifLuaTeX
\usepackage[bidi=basic]{babel}
\else
\usepackage[bidi=default]{babel}
\fi
\babelprovide[main,import]{ngerman}
% get rid of language-specific shorthands (see #6817):
\let\LanguageShortHands\languageshorthands
\def\languageshorthands#1{}
\ifLuaTeX
  \usepackage{selnolig}  % disable illegal ligatures
\fi
\IfFileExists{bookmark.sty}{\usepackage{bookmark}}{\usepackage{hyperref}}
\IfFileExists{xurl.sty}{\usepackage{xurl}}{} % add URL line breaks if available
\urlstyle{same} % disable monospaced font for URLs
\hypersetup{
  pdftitle={Zusammenfassung Advanced Embedded Systems},
  pdfauthor={Joel von Rotz \& Andreas Ming},
  pdflang={de},
  colorlinks=true,
  linkcolor={blue},
  filecolor={Maroon},
  citecolor={Blue},
  urlcolor={Blue},
  pdfcreator={LaTeX via pandoc}}

\title{Zusammenfassung Advanced Embedded Systems}
\author{Joel von Rotz \& Andreas Ming}
\date{23.02.23}

\begin{document}
 % [START] title
 % [ELSE] beamer

\makeatletter
\begin{center}
  \vspace*{0.5cm}
  
  \textbf{\Huge \@title}
  
  \vspace{0.1cm}

  {\Large {\@author \hspace{4.8cm} \@date}}
  
  \vspace{0.5cm}

\end{center}
\makeatother

 % [START] Source
\begin{center}
{\large \faGithub\space \href{https://www.youtube.com/watch?v=dQw4w9WgXcQ}{Quelldateien}}
\end{center}
 % [END] title


 % [END] beamer
 % [END] title


\ifdefined\Shaded\renewenvironment{Shaded}{\begin{tcolorbox}[colback={shadecolor}, boxrule=0pt, frame hidden, enhanced, breakable]}{\end{tcolorbox}}\fi\ifdefined\Shaded\renewenvironment{Shaded}{\begin{tcolorbox}[frame hidden, enhanced, breakable, colback={shadecolor}, boxrule=0pt]}{\end{tcolorbox}}\fi

\renewcommand*\contentsname{Inhaltsverzeichnis}
{
\hypersetup{linkcolor=}
\setcounter{tocdepth}{3}
\tableofcontents
}
\hypertarget{einfuxfchrung}{%
\section{Einführung}\label{einfuxfchrung}}

\hypertarget{architektur}{%
\section{Architektur}\label{architektur}}

\hypertarget{entwicklung}{%
\section{Entwicklung}\label{entwicklung}}

\hypertarget{firmware}{%
\section{Firmware}\label{firmware}}

\hypertarget{real-time-operating-system}{%
\section{\texorpdfstring{\textbf{R}eal \textbf{T}ime \textbf{O}perating
\textbf{S}ystem}{Real Time Operating System}}\label{real-time-operating-system}}

\hypertarget{kernel}{%
\section{Kernel}\label{kernel}}

\hypertarget{synchronisation}{%
\section{Synchronisation}\label{synchronisation}}

\hypertarget{nachrichten}{%
\section{Nachrichten}\label{nachrichten}}

\hypertarget{parallelituxe4t}{%
\section{Parallelität}\label{parallelituxe4t}}

\hypertarget{benutzer}{%
\section{Benutzer}\label{benutzer}}

\hypertarget{grafik}{%
\section{Grafik}\label{grafik}}

\hypertarget{fragen}{%
\section{Fragen}\label{fragen}}

\hypertarget{sw01-einfuxfchrung}{%
\subsection{SW01 Einführung}\label{sw01-einfuxfchrung}}

\hypertarget{administratives}{%
\subsubsection{Administratives}\label{administratives}}

\begin{enumerate}
\def\labelenumi{\arabic{enumi}.}
\tightlist
\item
  Was bedeutet \emph{Embedded}?
\end{enumerate}

\emph{Embedded} bedeutet `eingebettet' und weist darauf hin, dass ein
Komponente oder Objekt Teil eines Ganzen ist.

\begin{enumerate}
\def\labelenumi{\arabic{enumi}.}
\setcounter{enumi}{1}
\tightlist
\item
  Was ist ein \emph{Embedded System}?
\end{enumerate}

\emph{Embedded Systems} bestehen aus Rechner (zum Beispiel
Mikrocontroller), welche in einem grösseren System `eingebettet' sind,
also Teil eines Ganzen. Obwohl diese Rechner sind, sind diese nicht als
`normale' Rechner wie ein Desktop Computer oder Laptop erkennbar. Diese
besitzen keine typischen Merkmale, wie Bildschirm, Maus, Tastatur,
sondern sind für einen speziellen Zweck optimiert. Durch die
Spezialisierung besitzt meistens ein Embedded System beschränkte
Ressourcen oder sind auf gewisse Faktoren wie Grösse, Kosten oder
Energieverbrauch ausgelegt.

\begin{enumerate}
\def\labelenumi{\arabic{enumi}.}
\setcounter{enumi}{2}
\tightlist
\item
  Was bedeutet \emph{IPC}?
\end{enumerate}

\textbf{I}nter-\textbf{P}rocess \textbf{C}ommunication, wie zum Beispiel
eine Queue. Dies beschreibt die Kommunikation zwischen zwei Prozessen in
einem RTOS.

\begin{enumerate}
\def\labelenumi{\arabic{enumi}.}
\setcounter{enumi}{3}
\tightlist
\item
  Beurteile, ob ein Raspberry Pi ein \emph{Embedded System} ist.
\end{enumerate}

Ein Raspberry Pi kann für beide Fälle verwendet werden, entweder als ein
Modul in einem System, welche zum Beispiel Bildverarbeitung für ein
Bilderkennungssystem macht, oder als Desktop Computer verwendet werden.
Der Übergang ist fliessend. Man kann es als Mini-Computer bezeichnen.

\begin{enumerate}
\def\labelenumi{\arabic{enumi}.}
\setcounter{enumi}{4}
\tightlist
\item
  Erkläre, was man unter \emph{Build Tools} versteht.
\end{enumerate}

Unter Build Tools versteht man die Werkzeuge, welche das Programm in den
Maschinen Code übersetzt. Funktionen sind: Compiler, Linker, Standard
Libraries, Debugger und zusätzliche Tools. Das Build Environment
übernimmt die Anwendung dieser Werkzeuge.

\hypertarget{software-tools}{%
\subsubsection{Software \& Tools}\label{software-tools}}

\begin{enumerate}
\def\labelenumi{\arabic{enumi}.}
\tightlist
\item
  Eclipse ist eine sehr universell einsetzbare IDE, was vielleicht auch
  problematisch sein kann. Was wären mögliche Kritikpunkte?
\end{enumerate}

Weniger Performant, da es in Java geschrieben ist. Die Komplexität von
Eclipse führt ebenfalls zu Performance-Einbussen.

\begin{enumerate}
\def\labelenumi{\arabic{enumi}.}
\setcounter{enumi}{1}
\tightlist
\item
  Was hat wohl wesentlich dazu beigetragen, dass Eclipse als Open Source
  Projekt erfolgreich wurde?
\end{enumerate}

Durch das Eclipse Open Source Konsortium wurde die Entwicklung von
Eclipse stark gefördert und in die Öffentlichkeit gebracht. Ebenfalls
hat es das Plugin-System eingeführt.

\begin{enumerate}
\def\labelenumi{\arabic{enumi}.}
\setcounter{enumi}{2}
\tightlist
\item
  Was ist die Rolle einer Foundation wie die der für Eclipse? Inwiefern
  unterscheidet sich eine Foundation von einer Firma wie IBM?
\end{enumerate}

\emph{Foundation}s sind non-profit Organisationen.

\begin{enumerate}
\def\labelenumi{\arabic{enumi}.}
\setcounter{enumi}{3}
\tightlist
\item
  Hersteller bieten oft `Eval Boards' an (ähnlich wie das tinyK22). Was
  ist der Sinn und Zweck davon?
\end{enumerate}

Ein Eval-Board besitzt das Minimum der kritischen Komponenten, damit der
Haupt-Komponenten (z.B. Sensor) funktioniert. Es wird zum
Evaluieren/Austesten der Haupt-Komponent verwendet, aber auch für die
direkte Hardware Integration. Ebenfalls reduziert es den
Hardware-Designaufwand, da man keine eigene Eval-Boards machen muss.

\begin{enumerate}
\def\labelenumi{\arabic{enumi}.}
\setcounter{enumi}{4}
\tightlist
\item
  Welche Komponenten finden Sie typischerweise in einem SDK?
\end{enumerate}

Beispiel-Projekte, Dokumentation, Treiber, Lizenzinformationen.

\begin{enumerate}
\def\labelenumi{\arabic{enumi}.}
\setcounter{enumi}{5}
\tightlist
\item
  Das Raspberry Pi ist weder das beste, schnellste, modernste noch das
  billigste Board, trotzdem ist es ein Erfolg. Was könnten die
  Erfolgsfaktoren sein?
\end{enumerate}

Die Raspberry Foundation legt grossen Wert auf die Dokumentationen ihrer
Geräte/Produkte. Die Datenblätter \& andere Dokumentation sind sehr
detailliert beschrieben und besteht ebenfalls aus vielen Anleitungen,
wie man den Raspberry Pi.

\begin{enumerate}
\def\labelenumi{\arabic{enumi}.}
\setcounter{enumi}{6}
\tightlist
\item
  Was ist der Grund, dass man im Pins Tool für einen Pin einen
  Identifier verwendet?
\end{enumerate}

Damit im Code nicht mit \emph{Magic Numbers} gearbeitet wird, also im
Sinne dass man direkt sieht, mit welchem Pin man es zu tun hat. Man gibt
dem Pin eine Bedeutung.

\begin{enumerate}
\def\labelenumi{\arabic{enumi}.}
\setcounter{enumi}{7}
\tightlist
\item
  Sie wollen von Ihrer Firma ein Projekt in die Open Source Domäne
  `entlassen'. Was müssten Sie dabei berücksichtigen, damit es ein
  Erfolg wird?
\end{enumerate}

Alle verwendeten Software-Komponenten müssen öffentlich zugänglich sein
(nicht closed source) und die entsprechende Lizenz muss einer Open
Source Lizenz entsprechen (z.B. GNU GPLv3, MIT, Creative Commons).

\begin{enumerate}
\def\labelenumi{\arabic{enumi}.}
\setcounter{enumi}{8}
\tightlist
\item
  Sie realisieren ein neues Embedded System: Was ist der Unterschied
  zwischen \emph{Design} und \emph{Architektur}?
\end{enumerate}

Die Architektur beschreibt die Beziehung zwischen den Komponenten (z.B.
Kommunikation) und Design beschreibt die Implementation bezüglich der
Architektur.

\begin{enumerate}
\def\labelenumi{\arabic{enumi}.}
\setcounter{enumi}{9}
\tightlist
\item
  Für welche Anforderungen oder Anwendungen eignet sich eher ein FPGA
  als ein Mikrocontroller? Was sind die Gründe dafür?
\end{enumerate}

FPGAs werden für datenintensive, parallele und reaktionsschnelle Systeme
verwendet. Beispiel wäre ein Digital Oszilloskop.

\begin{enumerate}
\def\labelenumi{\arabic{enumi}.}
\setcounter{enumi}{8}
\tightlist
\item
  Was versteht man unter einer Debug Probe?
\end{enumerate}

Eine Debug-Probe ist ein Gerät, welches die Verbindung zwischen Target
und Host. Unter anderem, verwenden diese Geräte als Schnittstellen JTAG,
SWD oder ISP.

\begin{enumerate}
\def\labelenumi{\arabic{enumi}.}
\setcounter{enumi}{11}
\tightlist
\item
  Was ist \emph{CMSIS-DAP}?
\end{enumerate}

\textbf{C}ommon \textbf{M}icrocontroller \textbf{S}oftware
\textbf{I}ndustry \textbf{S}tandard - \textbf{D}ebug \textbf{A}ccess
\textbf{P}ort

CMSIS-DAP ist ein standardisiertes Kommunikationsprotokoll für
Debug-Zwecken. Die Firmware von CMSIS-DAP wird DAPLink verwendet.

\begin{enumerate}
\def\labelenumi{\arabic{enumi}.}
\setcounter{enumi}{12}
\tightlist
\item
  Was ist \emph{CMSIS}?
\end{enumerate}

\textbf{C}ommon \textbf{M}icrocontroller \textbf{S}oftware
\textbf{I}ndustry \textbf{S}tandard ist eine Kollektion von
verschiedenen Industrie-Standards für Mikrocontroller Systemen. Es
bietet Schnittstellen zu Prozessoren und Peripheriegeräten,
Echtzeitbetriebssystemen und Middleware-Komponenten. CMSIS umfasst einen
Liefermechanismus für Geräte, Platinen und Software und ermöglicht die
Kombination von Softwarekomponenten verschiedener Anbieter.

\hypertarget{sw02-architektur}{%
\subsection{SW02 Architektur}\label{sw02-architektur}}

\hypertarget{system}{%
\subsubsection{System}\label{system}}

\begin{enumerate}
\def\labelenumi{\arabic{enumi}.}
\item
  Nenne drei gute Beispiele eines transformierenden Systems.
\item
  Nenne drei gute Beispiele eines reaktiven Systems.
\item
  Nenne drei gute Beispiele eines interaktiven Systems.
\item
  Inwiefern unterscheiden sich transformierende Systeme von reaktiven
  Systemen?
\item
  Beschreibe ein gutes Beispiel eines transformierenden Systems, welches
  über einen Eingabestrom und zwei Ausgabeströme verfügt.
\item
  Gibt es ein Beispiel eines Embedded Systems ohne
  Benutzerschnittstelle?
\item
  Zu welcher System Klasse gehört ein `Embedded System'?
\item
  Viele Systeme sind eine Kombination von transformierenden, reaktiven
  und interaktiven Systemen. Bestimme diese am Beipsiel eines Smartphone
\item
  Wieso ist die Verarbeitungsqualität für transformierende Systeme so
  wichtig?
\item
  Wieso sind transformierende Systeme typischerweise optimiert für eine
  optimale Systemausnutzung?
\item
  Interaktive Systeme sind typischerweise optimiert für eine schnelle
  Antwortzeit. überlege typische Antwortzeiten für interaktive Systeme
  geben: Wovon hängen diese ab?
\item
  Was bedeutet `Verarbeitungsqualität' bei einem Audio Encoder System?
\item
  Klassifiziere die folgenden Systeme nach reaktiv, interaktiv und
  transformativ: Digital-Uhr, Airbag, Polizei-Radar, Feuer Alarmsystem,
  Geldautomat, Tankanzeige im Flugzeug.
\item
  Nenne einige Systeme, welche keinen Computer oder Mikroprozessor
  verwenden.
\item
  Mit mehr Speicher k¨onnen Systeme oft schneller rechnen. Probiere
  Beispiele dazu zu finden.
\end{enumerate}

\hypertarget{rechner}{%
\subsubsection{Rechner}\label{rechner}}

\begin{enumerate}
\def\labelenumi{\arabic{enumi}.}
\item
  Ist ein Intel basiertes Notebook eher eine von Neumann oder Harvard
  Architektur?
\item
  Ist der Instruktionssatz des tinyK22 CISC oder RISC?
\item
  Wie kann eine RISC Architektur einen Rechner beschleunigen, da doch
  dabei mehr Instruktionen ausgeführt werden müssen?
\item
  Wieso eignen sich SIMD Instruktionen vor allem für Signalverarbeitung?
\item
  Was sind die Grenzen eines SoC Ansatzes, und wie können diese
  überwunden werden?
\item
  Wieso benötigt man ein XiP Verfahren bei einem externen
  Programmspeicher? Hinweis: Adressbereiche.
\end{enumerate}

\hypertarget{cortex}{%
\subsubsection{Cortex}\label{cortex}}

\begin{enumerate}
\def\labelenumi{\arabic{enumi}.}
\item
  Eine Anwendung verwenden viele 32bit Multiplikationen und Divisionen.
  Eignet sich ein ARM Cortex-M0+ dafür? Was sind Alternativen?
\item
  Was ist der Grund, dass beim M7 oft ein externer Speicher zum Einsatz
  kommt?
\item
  Wieso ist eine MMU für den Einsatz eines Linux nötig?
\item
  Wieso wurde ARM mit den ARM11 so erfolgreich?
\item
  Der M3 war und ist sehr erfolgreich. Was waren die Gründe für den M0
  und M4?
\item
  Was ist das Konzept von TrustZone?
\item
  Was ist der Unterschied zwischen einer MPU und einer MMU?
\item
  Bringe ein Beispiel für eine Sättigungsarithmethik.
\item
  Welche Schlüsseleigenschaften wurden von welchen Firmen in die
  Gründung des ARM Joint Venture eingebracht und von wem?
\item
  Wieso wurde die mögliche Übernahme von ARM durch Nvidia kontrovers
  diskutiert?
\item
  Was sind die wichtigsten Erfolgsfaktoren von ARM Prozessoren aus Sicht
  der Anwender?
\item
  Was könnte ein guter Kritikpunkt an ARM und deren Prozessoren sein?
  Gibt es Alternativen?
\item
  RISC-V ist in `aller Munde': Beschreibe in ein paar kurzen Sätzen was
  RISC-V ist.
\item
  Welcher Vorteil hat Arm gegenüber einer Konkurrenz wie Intel?
\item
  Ein Temperatur Sensor unterstützt einen Temperaturbereich von -45 Grad
  Celsius bis 125 Grad Celsius mit einer Auflösung von 0.1 Grad? Ist
  dafür eine Gleitkomma-Repräsentation mit den zugehörigen Operationen
  angebracht? Was wäre eine Möglichkeit?
\end{enumerate}

\hypertarget{sw03-entwicklung}{%
\subsection{SW03 Entwicklung}\label{sw03-entwicklung}}

\hypertarget{prozess}{%
\subsubsection{Prozess}\label{prozess}}

\begin{enumerate}
\def\labelenumi{\arabic{enumi}.}
\item
  Was ist der Unterschied zwischen Verifikation und Validierung?
  Erläutere es mit einem Beispiel.
\item
  Erkläre den Unterschied zwischen Unit Test, Integration Test und
  System Test?
\item
  Wieso geht die Phase der Ausserbetriebnahme eines Produktes oft
  vergessen? Was sind mögliche Konsequenzen?
\item
  Das Wasserfall Modell wird oft als `schlecht' dargestellt? Ist das
  berechtigt?
\item
  Welche grundlegenden Vorteile führt das V Modell gegenüber dem
  Wasserfall Modell ein?
\item
  Welche Anforderungen stellt das Agile Modell an das Entwicklungsteam?
\item
  Unter welcher Annahme wird das Agile Modell nur einmal durchlaufen?
\item
  Wieso braucht es beim Agile Modell ein Backlog?
\end{enumerate}

\hypertarget{werkzeuge}{%
\subsubsection{Werkzeuge}\label{werkzeuge}}

\begin{enumerate}
\def\labelenumi{\arabic{enumi}.}
\item
  Was versteht man unter einem Refactoring und was ist das Ziel davon?
\item
  Was ist der Unterschied zwischen Coverage und Profiling?
\item
  Was ist der grosse Vorteil von statischen Analyse Werkzeugen gegenüber
  den dynamischen?
\item
  Welcher Vorteil ergibt eine Commit Phase mit einem VCS?
\item
  Gib ein Beispiel, wo es schwierig ist eine 100\% Coverage zu
  erreichen?
\end{enumerate}

\hypertarget{sw04-firmware}{%
\subsection{SW04 Firmware}\label{sw04-firmware}}

\hypertarget{architektur-1}{%
\subsubsection{Architektur}\label{architektur-1}}

\hypertarget{module}{%
\subsubsection{Module}\label{module}}

\hypertarget{bibliotheken}{%
\subsubsection{Bibliotheken}\label{bibliotheken}}

\hypertarget{sw05-rtos}{%
\subsection{SW05 RTOS}\label{sw05-rtos}}

\hypertarget{echtzeit}{%
\subsubsection{Echtzeit}\label{echtzeit}}

\hypertarget{freertos}{%
\subsubsection{FreeRTOS}\label{freertos}}

\hypertarget{archtiektur}{%
\subsubsection{Archtiektur}\label{archtiektur}}

\hypertarget{kernel-api}{%
\subsubsection{Kernel API}\label{kernel-api}}

\hypertarget{tasks}{%
\subsubsection{Tasks}\label{tasks}}

\hypertarget{sw06-kernel}{%
\subsection{SW06 Kernel}\label{sw06-kernel}}

\hypertarget{interrupts}{%
\subsubsection{Interrupts}\label{interrupts}}

\hypertarget{visualisierung}{%
\subsubsection{Visualisierung}\label{visualisierung}}

\hypertarget{sw07-synchronisation}{%
\subsection{SW07 Synchronisation}\label{sw07-synchronisation}}

\hypertarget{synchronisierung}{%
\subsubsection{Synchronisierung}\label{synchronisierung}}

\hypertarget{freertos-interrupts}{%
\subsubsection{FreeRTOS \& Interrupts}\label{freertos-interrupts}}

\hypertarget{sw08-nachrichten}{%
\subsection{SW08 Nachrichten}\label{sw08-nachrichten}}

\hypertarget{queues}{%
\subsubsection{Queues}\label{queues}}

\hypertarget{timer}{%
\subsubsection{Timer}\label{timer}}

\hypertarget{sw09-parallelituxe4t}{%
\subsection{SW09 Parallelität}\label{sw09-parallelituxe4t}}

\hypertarget{reentrancy}{%
\subsubsection{Reentrancy}\label{reentrancy}}

\hypertarget{sema}{%
\subsubsection{Sema}\label{sema}}

\hypertarget{sw10-benutzer}{%
\subsection{SW10 Benutzer}\label{sw10-benutzer}}

\hypertarget{benutzerschnittstellen}{%
\subsubsection{Benutzerschnittstellen}\label{benutzerschnittstellen}}

\hypertarget{sw11-grafik}{%
\subsection{SW11 Grafik}\label{sw11-grafik}}

\hypertarget{graphical-user-interface}{%
\subsubsection{Graphical User
Interface}\label{graphical-user-interface}}



\end{document}
