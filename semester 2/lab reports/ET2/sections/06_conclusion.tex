\documentclass[../main.tex]{subfiles}

\begin{document}
\section{Schlussfolgerung}

Die Magnetfeld- \& Widerstandsmessungen trafen den Erwartungen zu, aber bei beiden Messungen gab es Abweichungen.

Der gemessene Widerstand der Spule wich vom berechneten Widerstand um ca. $10\%$ ab. Ein Teil der Abweichung könnte von der Berechnung der Drahtlänge sein. Es wurde mit gestapelten Ringen gerechnet, anstatt wie eine Spule. Oder es kann auch von der Toleranz her kommen. Der gemessene Wert selbst aber kommt sehr nahe dem berechnet nahe.

Bei der Messung von $B_{Mitte}$ wurde anstatt $\SI{0.25}{\milli\tesla}$ der Wert $\SI{0.28}{\milli\tesla}$ gemessen. Mit Abweichungen musste gerechnet werden, da die Realität von Messungen nicht eins zu eins der Theorie entspricht. Die Abweichung wird aber vernachlässigt, weil diese Messung eine experimentelle Validierung ist, welche die Gleichung \ref{equ:flussdichte} zeigt. Diese Messung hätte man eventuell besser messen können, wenn man zusätzlich ein leicht ferromagnetisches Material für das Innere der Spule verwenden würde.

Interessant aber war, dass am Rand der Spule $B_{Top}$ eine schwächere Flussdichte gemessen wurde, da der Magnetische Fluss sich ausdehnte und nicht mehr in einer homogenen Richtung floss. Ausserhalb der Spule wurde ein negativer Wert gemessen, welcher der Theorie stimmen würde, aber ob dieser Wert genau diesem Fluss entspricht ist nicht klar.
\end{document}
