% Options for packages loaded elsewhere
\PassOptionsToPackage{unicode}{hyperref}
\PassOptionsToPackage{hyphens}{url}
\PassOptionsToPackage{dvipsnames,svgnames,x11names}{xcolor}
%
\documentclass[
  10pt,
  a4paperpaper,
  DIV=11,
  numbers=noendperiod]{scrartcl}

\usepackage{amsmath,amssymb}
\usepackage{lmodern}
\usepackage{iftex}
\ifPDFTeX
  \usepackage[T1]{fontenc}
  \usepackage[utf8]{inputenc}
  \usepackage{textcomp} % provide euro and other symbols
\else % if luatex or xetex
  \usepackage{unicode-math}
  \defaultfontfeatures{Scale=MatchLowercase}
  \defaultfontfeatures[\rmfamily]{Ligatures=TeX,Scale=1}
\fi
% Use upquote if available, for straight quotes in verbatim environments
\IfFileExists{upquote.sty}{\usepackage{upquote}}{}
\IfFileExists{microtype.sty}{% use microtype if available
  \usepackage[]{microtype}
  \UseMicrotypeSet[protrusion]{basicmath} % disable protrusion for tt fonts
}{}
\makeatletter
\@ifundefined{KOMAClassName}{% if non-KOMA class
  \IfFileExists{parskip.sty}{%
    \usepackage{parskip}
  }{% else
    \setlength{\parindent}{0pt}
    \setlength{\parskip}{6pt plus 2pt minus 1pt}}
}{% if KOMA class
  \KOMAoptions{parskip=half}}
\makeatother
\usepackage{xcolor}
\usepackage[top=15mm,bottom=25mm,left=15mm,right=15mm]{geometry}
\usepackage[normalem]{ulem}
\setlength{\emergencystretch}{3em} % prevent overfull lines
\setcounter{secnumdepth}{-\maxdimen} % remove section numbering
% Make \paragraph and \subparagraph free-standing
\ifx\paragraph\undefined\else
  \let\oldparagraph\paragraph
  \renewcommand{\paragraph}[1]{\oldparagraph{#1}\mbox{}}
\fi
\ifx\subparagraph\undefined\else
  \let\oldsubparagraph\subparagraph
  \renewcommand{\subparagraph}[1]{\oldsubparagraph{#1}\mbox{}}
\fi

\usepackage{color}
\usepackage{fancyvrb}
\newcommand{\VerbBar}{|}
\newcommand{\VERB}{\Verb[commandchars=\\\{\}]}
\DefineVerbatimEnvironment{Highlighting}{Verbatim}{commandchars=\\\{\}}
% Add ',fontsize=\small' for more characters per line
\newenvironment{Shaded}{}{}
\newcommand{\AlertTok}[1]{\textcolor[rgb]{1.00,0.33,0.33}{\textbf{#1}}}
\newcommand{\AnnotationTok}[1]{\textcolor[rgb]{0.42,0.45,0.49}{#1}}
\newcommand{\AttributeTok}[1]{\textcolor[rgb]{0.84,0.23,0.29}{#1}}
\newcommand{\BaseNTok}[1]{\textcolor[rgb]{0.00,0.36,0.77}{#1}}
\newcommand{\BuiltInTok}[1]{\textcolor[rgb]{0.84,0.23,0.29}{#1}}
\newcommand{\CharTok}[1]{\textcolor[rgb]{0.01,0.18,0.38}{#1}}
\newcommand{\CommentTok}[1]{\textcolor[rgb]{0.42,0.45,0.49}{#1}}
\newcommand{\CommentVarTok}[1]{\textcolor[rgb]{0.42,0.45,0.49}{#1}}
\newcommand{\ConstantTok}[1]{\textcolor[rgb]{0.00,0.36,0.77}{#1}}
\newcommand{\ControlFlowTok}[1]{\textcolor[rgb]{0.84,0.23,0.29}{#1}}
\newcommand{\DataTypeTok}[1]{\textcolor[rgb]{0.84,0.23,0.29}{#1}}
\newcommand{\DecValTok}[1]{\textcolor[rgb]{0.00,0.36,0.77}{#1}}
\newcommand{\DocumentationTok}[1]{\textcolor[rgb]{0.42,0.45,0.49}{#1}}
\newcommand{\ErrorTok}[1]{\textcolor[rgb]{1.00,0.33,0.33}{\underline{#1}}}
\newcommand{\ExtensionTok}[1]{\textcolor[rgb]{0.84,0.23,0.29}{\textbf{#1}}}
\newcommand{\FloatTok}[1]{\textcolor[rgb]{0.00,0.36,0.77}{#1}}
\newcommand{\FunctionTok}[1]{\textcolor[rgb]{0.44,0.26,0.76}{#1}}
\newcommand{\ImportTok}[1]{\textcolor[rgb]{0.01,0.18,0.38}{#1}}
\newcommand{\InformationTok}[1]{\textcolor[rgb]{0.42,0.45,0.49}{#1}}
\newcommand{\KeywordTok}[1]{\textcolor[rgb]{0.84,0.23,0.29}{#1}}
\newcommand{\NormalTok}[1]{\textcolor[rgb]{0.14,0.16,0.18}{#1}}
\newcommand{\OperatorTok}[1]{\textcolor[rgb]{0.14,0.16,0.18}{#1}}
\newcommand{\OtherTok}[1]{\textcolor[rgb]{0.44,0.26,0.76}{#1}}
\newcommand{\PreprocessorTok}[1]{\textcolor[rgb]{0.84,0.23,0.29}{#1}}
\newcommand{\RegionMarkerTok}[1]{\textcolor[rgb]{0.42,0.45,0.49}{#1}}
\newcommand{\SpecialCharTok}[1]{\textcolor[rgb]{0.00,0.36,0.77}{#1}}
\newcommand{\SpecialStringTok}[1]{\textcolor[rgb]{0.01,0.18,0.38}{#1}}
\newcommand{\StringTok}[1]{\textcolor[rgb]{0.01,0.18,0.38}{#1}}
\newcommand{\VariableTok}[1]{\textcolor[rgb]{0.89,0.38,0.04}{#1}}
\newcommand{\VerbatimStringTok}[1]{\textcolor[rgb]{0.01,0.18,0.38}{#1}}
\newcommand{\WarningTok}[1]{\textcolor[rgb]{1.00,0.33,0.33}{#1}}

\providecommand{\tightlist}{%
  \setlength{\itemsep}{0pt}\setlength{\parskip}{0pt}}\usepackage{longtable,booktabs,array}
\usepackage{calc} % for calculating minipage widths
% Correct order of tables after \paragraph or \subparagraph
\usepackage{etoolbox}
\makeatletter
\patchcmd\longtable{\par}{\if@noskipsec\mbox{}\fi\par}{}{}
\makeatother
% Allow footnotes in longtable head/foot
\IfFileExists{footnotehyper.sty}{\usepackage{footnotehyper}}{\usepackage{footnote}}
\makesavenoteenv{longtable}
\usepackage{graphicx}
\makeatletter
\def\maxwidth{\ifdim\Gin@nat@width>\linewidth\linewidth\else\Gin@nat@width\fi}
\def\maxheight{\ifdim\Gin@nat@height>\textheight\textheight\else\Gin@nat@height\fi}
\makeatother
% Scale images if necessary, so that they will not overflow the page
% margins by default, and it is still possible to overwrite the defaults
% using explicit options in \includegraphics[width, height, ...]{}
\setkeys{Gin}{width=\maxwidth,height=\maxheight,keepaspectratio}
% Set default figure placement to htbp
\makeatletter
\def\fps@figure{htbp}
\makeatother

\usepackage{amssymb, amsmath, etoolbox}
\usepackage{multicol}
\usepackage[dvipsnames]{xcolor}
\definecolor{darkgreen}{HTML}{014f32}
\usepackage[german]{babel}

\usepackage{fontawesome5}
\setlength\columnsep{20pt}

\usepackage[utf8]{inputenc}

\usepackage[nodisplayskipstretch]{setspace}

\newenvironment{conditions}
  { % start-code
    \par\vspace{\abovedisplayskip}\noindent
    \begin{tabular}{>{$}c<{$} @{${}:{}$} l}
  }
  { % end-code
    \end{tabular}
    \par\vspace{\belowdisplayskip}
  }

% used to configure 
\makeatletter
\renewcommand*\env@matrix[1][*\c@MaxMatrixCols c]{%
  \hskip -\arraycolsep
  \let\@ifnextchar\new@ifnextchar
  \array{#1}}
\makeatother

\usepackage{lmodern}
\renewcommand{\familydefault}{\sfdefault}
\usepackage{cmbright}
\KOMAoption{captions}{tableheading}
\makeatletter
\@ifpackageloaded{tcolorbox}{}{\usepackage[many]{tcolorbox}}
\@ifpackageloaded{fontawesome5}{}{\usepackage{fontawesome5}}
\definecolor{quarto-callout-color}{HTML}{909090}
\definecolor{quarto-callout-note-color}{HTML}{0758E5}
\definecolor{quarto-callout-important-color}{HTML}{CC1914}
\definecolor{quarto-callout-warning-color}{HTML}{EB9113}
\definecolor{quarto-callout-tip-color}{HTML}{00A047}
\definecolor{quarto-callout-caution-color}{HTML}{FC5300}
\definecolor{quarto-callout-color-frame}{HTML}{acacac}
\definecolor{quarto-callout-note-color-frame}{HTML}{4582ec}
\definecolor{quarto-callout-important-color-frame}{HTML}{d9534f}
\definecolor{quarto-callout-warning-color-frame}{HTML}{f0ad4e}
\definecolor{quarto-callout-tip-color-frame}{HTML}{02b875}
\definecolor{quarto-callout-caution-color-frame}{HTML}{fd7e14}
\makeatother
\makeatletter
\makeatother
\makeatletter
\makeatother
\makeatletter
\@ifpackageloaded{caption}{}{\usepackage{caption}}
\AtBeginDocument{%
\ifdefined\contentsname
  \renewcommand*\contentsname{Table of contents}
\else
  \newcommand\contentsname{Table of contents}
\fi
\ifdefined\listfigurename
  \renewcommand*\listfigurename{List of Figures}
\else
  \newcommand\listfigurename{List of Figures}
\fi
\ifdefined\listtablename
  \renewcommand*\listtablename{List of Tables}
\else
  \newcommand\listtablename{List of Tables}
\fi
\ifdefined\figurename
  \renewcommand*\figurename{Figure}
\else
  \newcommand\figurename{Figure}
\fi
\ifdefined\tablename
  \renewcommand*\tablename{Table}
\else
  \newcommand\tablename{Table}
\fi
}
\@ifpackageloaded{float}{}{\usepackage{float}}
\floatstyle{ruled}
\@ifundefined{c@chapter}{\newfloat{codelisting}{h}{lop}}{\newfloat{codelisting}{h}{lop}[chapter]}
\floatname{codelisting}{Listing}
\newcommand*\listoflistings{\listof{codelisting}{List of Listings}}
\makeatother
\makeatletter
\@ifpackageloaded{caption}{}{\usepackage{caption}}
\@ifpackageloaded{subcaption}{}{\usepackage{subcaption}}
\makeatother
\makeatletter
\@ifpackageloaded{tcolorbox}{}{\usepackage[many]{tcolorbox}}
\makeatother
\makeatletter
\@ifundefined{shadecolor}{\definecolor{shadecolor}{rgb}{.97, .97, .97}}
\makeatother
\makeatletter
\makeatother
\ifLuaTeX
  \usepackage{selnolig}  % disable illegal ligatures
\fi
\IfFileExists{bookmark.sty}{\usepackage{bookmark}}{\usepackage{hyperref}}
\IfFileExists{xurl.sty}{\usepackage{xurl}}{} % add URL line breaks if available
\urlstyle{same} % disable monospaced font for URLs
\hypersetup{
  pdftitle={Zusammenfassung Advanced Programming},
  pdfauthor={Joel von Rotz},
  colorlinks=true,
  linkcolor={blue},
  filecolor={Maroon},
  citecolor={Blue},
  urlcolor={Blue},
  pdfcreator={LaTeX via pandoc}}

\title{Zusammenfassung Advanced Programming}
\author{Joel von Rotz}
\date{19.06.1932}

\begin{document}
\maketitle
\ifdefined\Shaded\renewenvironment{Shaded}{\begin{tcolorbox}[breakable, boxrule=0pt, frame hidden, enhanced, sharp corners, borderline west={3pt}{0pt}{shadecolor}, interior hidden]}{\end{tcolorbox}}\fi

\renewcommand*\contentsname{Table of contents}
{
\hypersetup{linkcolor=}
\setcounter{tocdepth}{3}
\tableofcontents
}
\begin{multicols}{2}

\hypertarget{vergleich-c-c}{%
\section{\texorpdfstring{Vergleich \texttt{C} \&
\texttt{C\#}}{Vergleich C \& C\#}}\label{vergleich-c-c}}

\begin{itemize}
\tightlist
\item
  Jede Funktion muss zu einer Klasse gehören. Es gibt keine
  \uline{``nackten''} Funktionen
\end{itemize}

\hypertarget{datentypen}{%
\section{Datentypen}\label{datentypen}}

\hypertarget{string}{%
\subsection{String}\label{string}}

Strings werden mit dem folgender Deklaration \texttt{"inhalt"}

\begin{tcolorbox}[enhanced jigsaw, bottomrule=.15mm, title=\textcolor{quarto-callout-important-color}{\faExclamation}\hspace{0.5em}{Important}, toprule=.15mm, colback=white, coltitle=black, colbacktitle=quarto-callout-important-color!10!white, titlerule=0mm, toptitle=1mm, bottomtitle=1mm, colframe=quarto-callout-important-color-frame, leftrule=.75mm, rightrule=.15mm, opacityback=0, breakable, left=2mm, opacitybacktitle=0.6, arc=.35mm]

Strings können nicht verändert werden -\textgreater{} sind
\textbf{read-only}

\begin{Shaded}
\begin{Highlighting}[]
\DataTypeTok{string}\NormalTok{ s }\OperatorTok{=} \StringTok{"Hallo Welt"}\OperatorTok{;}

\NormalTok{s}\OperatorTok{[}\DecValTok{1}\OperatorTok{]} \OperatorTok{=} \CharTok{\textquotesingle{}A\textquotesingle{}}\OperatorTok{;} \CommentTok{// ERROR}
\end{Highlighting}
\end{Shaded}

\end{tcolorbox}

\hypertarget{parameter-in-string-einfuxfcgen}{%
\subsubsection{Parameter in String
einfügen}\label{parameter-in-string-einfuxfcgen}}

Parameter/variablen können in Strings direkt eingefügt werden.

\end{multicols}

\hypertarget{bildschirmausgabe}{%
\subsubsection{Bildschirmausgabe}\label{bildschirmausgabe}}

\textbf{Variante 1} - C Style:

\begin{Shaded}
\begin{Highlighting}[]
\NormalTok{Console}\OperatorTok{.}\FunctionTok{WriteLine}\OperatorTok{(}\StringTok{"The sum of \{0\} and \{1\} is \{2\}"}\OperatorTok{,}\NormalTok{a}\OperatorTok{,}\NormalTok{b}\OperatorTok{,}\NormalTok{result}\OperatorTok{);}
\end{Highlighting}
\end{Shaded}

\textbf{Variante 2} - C\# Style:

\begin{Shaded}
\begin{Highlighting}[]
\NormalTok{Console}\OperatorTok{.}\FunctionTok{WriteLine}\OperatorTok{(}\StringTok{"The sum of"} \OperatorTok{+}\NormalTok{ a }\OperatorTok{+} \StringTok{"and"} \OperatorTok{+}\NormalTok{ b }\OperatorTok{+} \StringTok{"is"} \OperatorTok{+}\NormalTok{ result}\OperatorTok{);}
\end{Highlighting}
\end{Shaded}

\textbf{Variante 3} - new C\# Style:

\begin{Shaded}
\begin{Highlighting}[]
\NormalTok{Console}\OperatorTok{.}\FunctionTok{WriteLine}\OperatorTok{(}\NormalTok{$}\StringTok{"The sum of \{a\} and \{b\} is \{result\}"}\OperatorTok{);}
\end{Highlighting}
\end{Shaded}

\begin{multicols}{2}

\hypertarget{overloading}{%
\section{Overloading}\label{overloading}}

\hypertarget{konstruktor-overloading}{%
\subsection{Konstruktor Overloading}\label{konstruktor-overloading}}

\begin{Shaded}
\begin{Highlighting}[]
\KeywordTok{class}\NormalTok{ Point }\OperatorTok{\{}
  \KeywordTok{private} \DataTypeTok{int}\NormalTok{ pos\_x}\OperatorTok{;}
  \KeywordTok{private} \DataTypeTok{int}\NormalTok{ pos\_y}\OperatorTok{;}

  \KeywordTok{public} \FunctionTok{Point}\OperatorTok{(}\DataTypeTok{int}\NormalTok{ x}\OperatorTok{,} \DataTypeTok{int}\NormalTok{ y}\OperatorTok{)} \OperatorTok{\{}
    \KeywordTok{this}\OperatorTok{.}\FunctionTok{pos\_x} \OperatorTok{=}\NormalTok{ x}\OperatorTok{;}
    \KeywordTok{this}\OperatorTok{.}\FunctionTok{pos\_y} \OperatorTok{=}\NormalTok{ y}\OperatorTok{;}
  \OperatorTok{\}}

  \KeywordTok{public} \FunctionTok{Point}\OperatorTok{()} \OperatorTok{:} \KeywordTok{this}\OperatorTok{(}\DecValTok{0}\OperatorTok{,}\DecValTok{0}\OperatorTok{)} \OperatorTok{\{\}}
\OperatorTok{\}}
\end{Highlighting}
\end{Shaded}

Mit \texttt{this} nach dem Konstruktor (unterteilt mit \texttt{:}) kann
der Aufruf auf einen anderen Konstruktor weitergeleitet werden. Der
Inhalt des vorherigen Konstruktors wird erst nach dem Ablauf des
\texttt{this}-Konstruktors (im Beispiel
\texttt{Point(int\ x,\ int\ y)}).

\hypertarget{konstruktor-aufruf-reihenfolge}{%
\subsubsection{Konstruktor
Aufruf-Reihenfolge}\label{konstruktor-aufruf-reihenfolge}}

\begin{Shaded}
\begin{Highlighting}[]
\KeywordTok{using}\NormalTok{ System}\OperatorTok{;}

\KeywordTok{class}\NormalTok{ Point }\OperatorTok{\{}
  \KeywordTok{private} \DataTypeTok{int}\NormalTok{ pos\_x}\OperatorTok{;}
  \KeywordTok{private} \DataTypeTok{int}\NormalTok{ pos\_y}\OperatorTok{;}

  \KeywordTok{public} \FunctionTok{Point}\OperatorTok{(}\DataTypeTok{int}\NormalTok{ x}\OperatorTok{,} \DataTypeTok{int}\NormalTok{ y}\OperatorTok{)} \OperatorTok{\{}
    \KeywordTok{this}\OperatorTok{.}\FunctionTok{pos\_x} \OperatorTok{=}\NormalTok{ x}\OperatorTok{;}
    \KeywordTok{this}\OperatorTok{.}\FunctionTok{pos\_y} \OperatorTok{=}\NormalTok{ y}\OperatorTok{;}
\NormalTok{    Console}\OperatorTok{.}\FunctionTok{WriteLine}\OperatorTok{(}\NormalTok{$}\StringTok{"Point \{this.pos\_x\}, \{this.pos\_y\}"}\OperatorTok{);}
  \OperatorTok{\}}

  \KeywordTok{public} \FunctionTok{Point}\OperatorTok{(}\DataTypeTok{int}\NormalTok{ x}\OperatorTok{)} \OperatorTok{:} \KeywordTok{this}\OperatorTok{(}\NormalTok{x}\OperatorTok{,} \DecValTok{0}\OperatorTok{)} \OperatorTok{\{}
\NormalTok{    Console}\OperatorTok{.}\FunctionTok{WriteLine}\OperatorTok{(}\StringTok{"x{-}only"}\OperatorTok{);}
  \OperatorTok{\}}

  \KeywordTok{public} \FunctionTok{Point}\OperatorTok{()} \OperatorTok{:} \KeywordTok{this}\OperatorTok{(}\DecValTok{0}\OperatorTok{,}\DecValTok{0}\OperatorTok{)} \OperatorTok{\{\}}
\NormalTok{  Console}\OperatorTok{.}\FunctionTok{WriteLine}\OperatorTok{(}\StringTok{"no value"}\OperatorTok{);}
\OperatorTok{\}}
\end{Highlighting}
\end{Shaded}

Wird nun \texttt{Point(4)} aufgerufen erhält man folgendes auf der
Konsole

\begin{Shaded}
\begin{Highlighting}[]
\NormalTok{Point 4, 0}
\NormalTok{x{-}only}
\end{Highlighting}
\end{Shaded}

\hypertarget{default-parameter-implizit-overloading}{%
\subsection{Default Parameter (implizit
Overloading)}\label{default-parameter-implizit-overloading}}

\begin{Shaded}
\begin{Highlighting}[]

\end{Highlighting}
\end{Shaded}

\hypertarget{funktion}{%
\section{Funktion}\label{funktion}}

\hypertarget{out}{%
\subsection{\texorpdfstring{\texttt{out}}{out}}\label{out}}

\begin{Shaded}
\begin{Highlighting}[]

\end{Highlighting}
\end{Shaded}

\hypertarget{notes}{%
\section{Notes}\label{notes}}

\hypertarget{overflows-integer}{%
\subsection{Overflows Integer}\label{overflows-integer}}

Im folgenden Code wird eine Variable \texttt{i} mit dem maximalen Wert
eines \texttt{int} geladen und folgend inkrementiert.

\begin{Shaded}
\begin{Highlighting}[]
\DataTypeTok{int}\NormalTok{ i }\OperatorTok{=} \DataTypeTok{int}\OperatorTok{.}\FunctionTok{MaxValue}\OperatorTok{;}
\NormalTok{i}\OperatorTok{++;}
\end{Highlighting}
\end{Shaded}

Wird aber dies direkt in der Initialisierung eingebettet
(\texttt{...+\ 1}), ruft der Compiler aus, da er den Overflow erkennt.

\begin{Shaded}
\begin{Highlighting}[]
\DataTypeTok{int}\NormalTok{ i }\OperatorTok{=} \DataTypeTok{int}\OperatorTok{.}\FunctionTok{MaxValue} \OperatorTok{+} \DecValTok{1}\OperatorTok{;} \CommentTok{// COMPILE{-}FEHLER}
\NormalTok{i}\OperatorTok{++;}
\end{Highlighting}
\end{Shaded}

\begin{tcolorbox}[enhanced jigsaw, bottomrule=.15mm, title=\textcolor{quarto-callout-caution-color}{\faFire}\hspace{0.5em}{Danger}, toprule=.15mm, colback=white, coltitle=black, colbacktitle=quarto-callout-caution-color!10!white, titlerule=0mm, toptitle=1mm, bottomtitle=1mm, colframe=quarto-callout-caution-color-frame, leftrule=.75mm, rightrule=.15mm, opacityback=0, breakable, left=2mm, opacitybacktitle=0.6, arc=.35mm]

Dieser Overflow-Fehler gilt nur bei \textbf{konstanten} Werten bei der
Initialisierung. Wird eine separate Variable mit dem Maximalwert
initialisierit und an \texttt{i} hinzuaddiert, gibt es keinen Fehler.

\begin{Shaded}
\begin{Highlighting}[]
\DataTypeTok{int}\NormalTok{ k }\OperatorTok{=} \DataTypeTok{int}\OperatorTok{.}\FunctionTok{MaxValue}\OperatorTok{;}
\DataTypeTok{int}\NormalTok{ i }\OperatorTok{=}\NormalTok{ k }\OperatorTok{+} \DecValTok{1}\OperatorTok{;} \CommentTok{// KEIN Fehler}
\end{Highlighting}
\end{Shaded}

\end{tcolorbox}

\end{multicols}



\end{document}
